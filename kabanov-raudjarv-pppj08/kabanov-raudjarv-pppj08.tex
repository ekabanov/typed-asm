%\documentclass[12pt,a4paper]{report}
%\documentclass{acm_proc_article-sp}
\documentclass{sig-alternate}

\usepackage{url}
\usepackage{graphics}
\usepackage{epsfig}
\usepackage[
  pdftex,
  colorlinks=true,
  linkcolor=black,
  citecolor=black,
  urlcolor=blue,
  bookmarksopen=true,
  pdfpagemode=UseOutlines,
  pdfstartview=FitH]{hyperref}

\sloppy

\begin{document}

\title{Embedded Typesafe Domain Specific Languages for Java} 

\numberofauthors{2}
\author{
	\alignauthor 
		Jevgeni Kabanov\\
       \affaddr{Dept. of Computer Science}\\
       \affaddr{University of Tartu}\\
       \affaddr{Liivi 2, Tartu, Estonia}\\
		\email{ekabanov@gmail.com}
	\alignauthor 
		Rein Raudj\"arv\\
       \affaddr{Dept. of Computer Science}\\
       \affaddr{University of Tartu}\\
       \affaddr{Liivi 2, Tartu, Estonia}\\
		\email{reinra@gmail.com}
}

\date{}

\maketitle

\begin{abstract}
Projects like jMock and Hibernate Criteria Query introduced embedded DSLs into Java. We describe two case studies in which we develop embedded typesafe DSLs for building SQL queries and engineering Java bytecode. We proceed to extract several patterns useful for developing typesafe DSLs for arbitrary domains. Unlike most previous Java DSLs we find that mixing the Fluent Interface idiom with static functions, metadata and closures provides for a better user experience than pure method chaining. We also make very liberal use of the Java 5 Generics to improve the type safety properties of the DSLs.
\end{abstract}

%\tableofcontents

%%%%%%%%%%%%%%%%%%%%%%%%%%%%%%%%%%%%%%%%%%%%%%%%%%%%%%%%%%%%%%%%%%%%%%%%%%%%%%%%
%%%%%%%%%%%%%%%%%%%%%%%%%%%%%%%%%%%%%%%%%%%%%%%%%%%%%%%%%%%%%%%%%%%%%%%%%%%%%%%%
\section{Introduction}
%%%%%%%%%%%%%%%%%%%%%%%%%%%%%%%%%%%%%%%%%%%%%%%%%%%%%%%%%%%%%%%%%%%%%%%%%%%%%%%%
%%%%%%%%%%%%%%%%%%%%%%%%%%%%%%%%%%%%%%%%%%%%%%%%%%%%%%%%%%%%%%%%%%%%%%%%%%%%%%%%

Domain specific language usually refers to a small sublanguage that has very little overhead when expressing domain specific data and behaviour. DSL is a broad term \cite{vandeursen2000dsl,bentley1999pp} and can refer both to a fully implemented language and a specialised API that looks like a sublanguage \cite{hudak1996}, but still written using some general-purpose language. Such DSLs in the latter meaning have been introduced by both the functional \cite{bringert2004sph} and dynamic language communities \cite{cuadrado2007bds}. Both these communities (especially functional) took advantage of function composition and operator overloading to build combinator-based languages that look nothing like the original. The functional community also strongly supports the notion of type safety; therefore DSLs they create are usually statically typed.

In Java the DSLs are also becoming increasingly popular. The first examples like jMock \cite{freeman2004jsr} and Hibernate Criteria \cite{bauer2005ha} introduced a technique for writing embedded DSLs using method call chaining that was coined Fluent Interface. 

In this paper we show that although Fluent Interface is a powerful concept it is not fitting in all contexts. We propose to mix it with static functions, metadata and closures to make full use of Java language capabilities. We also propose to make use of Java Generics to make the embedded DSLs truly typesafe. We test our proposals on two case studies---embedded DSLs for manipulating DSL queries and engineering Java bytecode.

The rest of the paper is organised as follows. Section 2 introduces the \emph{Typesafe SQL} DSL and studies how it implements different aspects of SQL. Section 3 introduces the \emph{Typesafe Bytecode Engineering} DSL and studies how stack and variables can be encoded in a typesafe manner. Section 4 studies and discusses the generic patterns introduced in the case studies. Sections 5 and 6 conclude the paper and discuss some possible directions for further work.

%%%%%%%%%%%%%%%%%%%%%%%%%%%%%%%%%%%%%%%%%%%%%%%%%%%%%%%%%%%%%%%%%%%%%%%%%%%%%%%%
%%%%%%%%%%%%%%%%%%%%%%%%%%%%%%%%%%%%%%%%%%%%%%%%%%%%%%%%%%%%%%%%%%%%%%%%%%%%%%%%
\section{Typesafe SQL}
%%%%%%%%%%%%%%%%%%%%%%%%%%%%%%%%%%%%%%%%%%%%%%%%%%%%%%%%%%%%%%%%%%%%%%%%%%%%%%%%
%%%%%%%%%%%%%%%%%%%%%%%%%%%%%%%%%%%%%%%%%%%%%%%%%%%%%%%%%%%%%%%%%%%%%%%%%%%%%%%%

Let's start with a very simple example of an SQL query in Java. 
\begin{verbatim}
ResultSet rs = SqlUtil.executeQuery(
  "SELECT name, height, birthday" + 
  "FROM person" +
  "WHERE height >= " + 170);
while (rs.next()) {
   String name = rs.getString("name");
   Integer height = rs.getInt("height");
   Date birthday = rs.getDate("birthday");
   System.out.println(
     name + " " + height + " " + birthday);
}
\end{verbatim}

Already this example, simple as it is, illustrates the mistakes we can make:
\begin{itemize}
\item We could have misspelled any SQL command, like ``SELETC''.
\item We could be misspelling the table name or the field names.
\item We could be mistaken about the field type (e.g. height being a \emph{boolean}, where \emph{true} means ``over 170 cm'').
\item We could be reading wrong types from the result set.
\end{itemize}
The problem is that we would only find out about those errors when the query is executed. To make it worse, some errors would not even be reported and since most queries are assembled dynamically we can't ever be sure that it is error free.

The solution we propose is to build on recent innovation in the area and embeds the whole of the SQL as a typesafe embedded DSL. The following example shows what we propose it to look like:

\begin{verbatim}
Person p = new Person(); 

List<Tuple3<String, Integer, Date>> rows =
  new QueryBuilder(datasource)
    .from(p)
    .where(gt(p.height, 170))
    .select(p.name, p.height, p.birthday)
    .list();
for (Tuple3<String, Integer, Date> row : rows) {
  String name = row.v1;
  Integer height = row.v2;
  Date birthday = row.v3;
  System.out.println(
    name + " " + height + " " + birthday);
}
\end{verbatim}

Although this example does the same as the previous there is not a single string and any kind of misspelling or type inconsistency will show immediately as a compile-time error\footnote{Even more importantly with a sufficiently advanced IDE it will be marked as an error directly in the text of the program providing immediate feedback.}. In the following sections we explain how to achieve full type safety in this DSL.

%%%%%%%%%%%%%%%%%%%%%%%%%%%%%%%%%%%%%%%%%%%%%%%%%%%%%%%%%%%%%%%%%%%%%%%%%%%%%%%%
\subsection{Tuples}
%%%%%%%%%%%%%%%%%%%%%%%%%%%%%%%%%%%%%%%%%%%%%%%%%%%%%%%%%%%%%%%%%%%%%%%%%%%%%%%%

You may have already noticed that we make sure the result set types are not inconsistent by combining them into a class called \verb!Tuple3!.

Tuples are sequence of values where each component of a tuple is a value of specified type. Often used in functional languages they are not natively supported in Java. All the same the corresponding classes can be easily generated. For example a tuple with the length of two is following:

\begin{verbatim}
public class Tuple2<T1, T2> implements Tuple {
  public final T1 v1;
  public final T2 v2;

  public Tuple2(T1 v1, T2 v2) {
    this.v1 = v1;
    this.v2 = v2;
  }		

  public T1 v1() { return v1; }
  public T2 v2() { return v2; }
}
\end{verbatim}

We use tuples to return the query results with the right types. Instead of \verb!Tuple1! we can just use the type itself.

%%%%%%%%%%%%%%%%%%%%%%%%%%%%%%%%%%%%%%%%%%%%%%%%%%%%%%%%%%%%%%%%%%%%%%%%%%%%%%%%
\subsection{Metadata Dictionary}
%%%%%%%%%%%%%%%%%%%%%%%%%%%%%%%%%%%%%%%%%%%%%%%%%%%%%%%%%%%%%%%%%%%%%%%%%%%%%%%%

The first step towards type safety of the query itself is ensuring that table and column names we use do in fact exist and are spelled correctly. To ensure that we use the database metadata about the tables and columns to generate a typesafe \emph{metadata dictionary}.

Metadata dictionary is a set of information about database describing tables and columns with their types. In Java the dictionary of the table Person could be the following\footnote{We do not care how the dictionary is generated as it is irrelevant to how it can be used}:

\begin{verbatim}
public class Person implements Table {
  public String getName() { return "person"; };

  public Column<Person, String> name = 
    new Column<Person, String>(
      this, "name", String.class);
  public Column<Person, Integer> height = 
    new Column<Person, Integer>(
      this, "height", Integer.class);
  public Column<Person, Date> birthday = 
    new Column<Person, Date>(
      this, "birthday", Date.class);
}
\end{verbatim}

This metadata dictionary for the \verb!Person! table associates the table with
its name and columns. Each column is in turn associated with its name, type and
owner table. The generic type variables in the column definition provide us
with compile-time type information.

%%%%%%%%%%%%%%%%%%%%%%%%%%%%%%%%%%%%%%%%%%%%%%%%%%%%%%%%%%%%%%%%%%%%%%%%%%%%%%%%
\subsection{Builders}
%%%%%%%%%%%%%%%%%%%%%%%%%%%%%%%%%%%%%%%%%%%%%%%%%%%%%%%%%%%%%%%%%%%%%%%%%%%%%%%%

To make use of the metadata we need to build the query itself. We proceed by separating the query building into stages (\emph{from}, \emph{where}, \emph{select},~\ldots) and delegating a \emph{builder} for each of those stages. Thus we make sure that the basic syntax of the query is always correct since mistakes result in compile-time error.

One of the main idioms in creating Java DSLs is hiding the return type by chaining the calls on the previous call result. Although typically most methods will return ``this'' we can use it to stage the query building and allow only relevant methods to be called.

To examine this in detail let's recall our previous example, but omit the ``where'' part for the moment:

\begin{verbatim}
  Person p = new Person();

  List<Tuple3<String, Integer, Date>> persons = 
    new QueryBuilder(datasource)
      .from(p)
      .select(p.name, p.height, p.birthday)
      .list();
\end{verbatim}


The \verb!QueryBuilder! does not do much more than store the datasource. The \verb!from()! method returns the \verb!FromBuilder! that stores the table from the dictionary:

\begin{verbatim}
public class QueryBuilder extends Builder {
  ...
  public <T extends Table> FromBuilder<T> 
    from(T table);
}
\end{verbatim}

The \verb!FromBuilder.select()! method returns \verb!SelectBuilder3! that stores three specific columns from the table. 

\begin{verbatim}
public class FromBuilder<T extends Table> 
  extends Builder {
  ...
  public <C1> SelectBuilder1<T, C1> 
    select(Column<T, C1> c1);
  public <C1, C2> SelectBuilder2<T, C1, C2> 
    select(Column<T, C1> c1, Column<T, C2> c2);
  public <C1, C2, C3> SelectBuilder3<T, C1, C2, C3> 
    select(
      Column<T, C1> c1, 
      Column<T, C2> c2, 
      Column<T, C3> c3);
  ...
}
\end{verbatim}

Note that instead of having only one \verb!SelectBuilder! we choose to have many of them, numbered according to the amount of columns selected. Each of them carries all of the selected column types as generic type variables. This type variables can be used to generate tupled result or use the query as a subquery in a \emph{from} clause.

Finally \verb!SelectBuilder3.list()! constructs the SQL query, executes it and returns the result:

\begin{verbatim}
public class
  SelectBuilder3<T extends Table,C1,C2,C3> 
    extends SelectBuilder<T> {
  ...
  public List<Tuple3<C1,C2,C3>> list();
} 
\end{verbatim}

Note that since our builders carry the type of the table passed in \verb!from()! and that the \verb!FromBuilder! only accepts the columns belonging to the same type. This provides additional safety as the programmer cannot select fields from a table that was not written in from.

However this solution is hard to extend when there is more than one table in the \emph{from} clause. We could apply the same idiom and tuple all the builders over the \emph{from} table types, but to actually check the type we need the methods to be indexed by the table type indexes (e.g. by writing \verb!select1()!, \verb!select2()!,~\ldots). Since this is uncomfortable and is influenced by changes in \emph{from} clause we decided to leave this check out altogether and the builders do not carry the table types in the actual implementation.

%%%%%%%%%%%%%%%%%%%%%%%%%%%%%%%%%%%%%%%%%%%%%%%%%%%%%%%%%%%%%%%%%%%%%%%%%%%%%%%%
\subsection{Expressions}
%%%%%%%%%%%%%%%%%%%%%%%%%%%%%%%%%%%%%%%%%%%%%%%%%%%%%%%%%%%%%%%%%%%%%%%%%%%%%%%%

Now that we have the basic structure of the SQL queries set we need to encode arbitrary functions, aggregates and expressions. In a usual fluent interface they would be accessible using the same chained notation we used for building the query. However we chose instead to use static methods, imported to the local namespace using the \verb!import static! feature introduced in Java 5.

As far as we care a general SQL expression can be expressed with the following interface:

\begin{verbatim}
public interface Expression<E> {
  String getSqlString();
  List<Object> getSqlArguments();
  Class<E> getType();		
}
\end{verbatim}

The \verb!E! type variable is the type of the value that the expression produces on evaluation. The corresponding class is returned by the \verb!getType()! method. Finally \verb!getSqlString()! returns the corresponding SQL fragment and \verb!getSqlArguments()! returns the arguments to be inserted instead of ``?'' in the query. 

In \emph{where} clause we only permit to use expressions of type \verb!Expression<Boolean>!\footnote{Since we would like to permit arbitrary many such expressions to be passed to methods the ``varargs'' feature introduced in Java 5 is used. However since array component types cannot be generic we have to introduce the \texttt{BooleanExpression extends Expression$<$Boolean$>$} which complicates things a bit. We ignore this complication in the examples.} such as "like", "$<$", "=", etc. The operands of these expressions can already be arbitrary. To create these expressions we could use the following API:

\begin{verbatim}
public class ExpressionUtil {
  public static <E> Expression<E> 
    constant(E value);

  public static <E> Expression<Boolean> 
    eq(Expression<E> e1, Expression<E> e2);

  public static <E> Expression<Boolean> 
    gt(Expression<E> e1, Expression<E> e2);

  public static Expression<Boolean> 
    like(Expression<?> e,
      Expression<String> pattern);

  public static Expression<Boolean> 
    not(Expression<Boolean> e);

  public static Expression<Boolean> 
    and(Expression<Boolean>... e);
  ...
}
\end{verbatim}

The \verb!constant()! method returns an expression corresponding to the value. Since we can overload all of the methods to also accept basic values and just call \verb!constant()! for them we will not call it explicitly in the upcoming examples.

Now that we introduced the \verb!Expression! type we can finally define the \verb!Column! type we used to encode column metadata:

\begin{verbatim}
public class Column<T extends Table, C> 
  implements Expression<C> {
  ...
  public Class<C> 
    getType() { return type; }
  public String 
    getSqlString() { return name; }
  public List<Object> 
    getSqlArguments() { return null; }		
}
\end{verbatim}

It is as straightforward as the constant expression and just returns the name of the column as the SQL fragment.

We can now easily encode the expression \verb!WHERE name = 'Peter' or height > 170!:

\begin{verbatim}
  Person p = new Person();
		
  List<Tuple3<String, Integer, Date>> persons =
    new QueryBuilder(datasource)
      .from(p)
      .where(or(
          eq(p.name, "Peter"),
          gt(p.height, 170))
        )
      .select(p.name, p.height, p.birthday)
      .list();
\end{verbatim}


So far we have only allowed to select columns from the table. In general we want to select an arbitrary expression (such as \verb!concat(name, ', ', birthday)!). Therefore the \verb!FromBuilder! class should just accept \verb!Expression!s instead of \verb!Column!s:

\begin{verbatim}
public class FromBuilder extends Builder {
  ...
  public <C1> SelectBuilder1<C1> 
    select(Expression<C1> c1);
  public <C1, C2> SelectBuilder2<C1, C2> 
    select(Expression<C1> c1, Expression<C2> c2);
  public <C1, C2, C3> SelectBuilder3<C1, C2, C3> 
    select(
      Expression<C1> c1, 
      Expression<C2> c2, 
      Expression<C3> c3);
  ...
}
\end{verbatim}

%%%%%%%%%%%%%%%%%%%%%%%%%%%%%%%%%%%%%%%%%%%%%%%%%%%%%%%%%%%%%%%%%%%%%%%%%%%%%%%%
\subsection{Aliases}
%%%%%%%%%%%%%%%%%%%%%%%%%%%%%%%%%%%%%%%%%%%%%%%%%%%%%%%%%%%%%%%%%%%%%%%%%%%%%%%%

Sometimes we need to have aliased SQL subexpressions that can be used in both \emph{select} and \emph{where} clauses. For this we introduce an \verb!Alias! class that can be used to hold such expressions in variables or fields. E.g. \texttt{SELECT concat(first\_name, " ", last\_name) as full\_name WHERE full\_name != "Peter Griffin"} would look like:

\begin{verbatim}
  Person p = new Person();
  Alias<String> fullName =
    alias(concat(p.firstName, " ", p.lastName));
		
  List<String> names = new QueryBuilder(datasource)
    .from(p)
    .where(not(eq(fullName, "Peter Griffin")))
    .select(fullName)
    .list();
\end{verbatim}

The \verb!Alias! class differs from \verb!Expression! by having a \verb!getAliasExpression()! method that returns the original expression appended by \verb!AS (new name)!. 

\begin{verbatim}
public interface Alias<E> extends Expression<E> {
  String getName();
  Expression<E> getAliasExpression();
}
\end{verbatim}
 
Another use of SQL aliases is to use same table more than once in a query. To do that we just create separate instances of the table class:

\begin{verbatim}
  Person person = new Person();
  Person father = new Person();
		
  List<Tuple2<String, String>> names =
    new QueryBuilder(datasource)
      .from(person, father)
      .where(eq(person.fatherId, father.id))
      .select(person.name, father.name)
      .list();
\end{verbatim}

%%%%%%%%%%%%%%%%%%%%%%%%%%%%%%%%%%%%%%%%%%%%%%%%%%%%%%%%%%%%%%%%%%%%%%%%%%%%%%%%
\subsection{Control Flow and Reuse}
%%%%%%%%%%%%%%%%%%%%%%%%%%%%%%%%%%%%%%%%%%%%%%%%%%%%%%%%%%%%%%%%%%%%%%%%%%%%%%%%

The next problem is how to mix the DSL with the general-purpose control flow and method calls. The problem here is that (although it is hidden from the user) we return a different type every time. To solve this we encourage to use closures for the control flow:

\begin{verbatim}
public interface Closure {
  void apply(Builder builder);
}

public class SelectBuilderC2<C1,C2> 
  extends SelectBuilder {
  ...
  public SelectBuilderC2<C1,C2> 
    closure(Closure closure) {

    closure.apply(this);
    return this;
  }
}
\end{verbatim}

So to find persons by name (and all persons if the name is \verb!null!) we can use the following syntax:

\begin{verbatim}
  Person p = new Person();

  List<Tuple2<Integer, String>> rows =
    new QueryBuilder(datasource)
      .from(p)
      .closure(new Closure() {
        public void apply(Builder builder) {
          if (searchName != null) {
            builder.addConditions(
              eq(p.name, searchName));
          }
        }
      })
      .select(p.id, p.name)
      .list();
\end{verbatim}

Such alternative API also allows to modify the query in reusable methods called using closures.

%%%%%%%%%%%%%%%%%%%%%%%%%%%%%%%%%%%%%%%%%%%%%%%%%%%%%%%%%%%%%%%%%%%%%%%%%%%%%%%%
\subsection{The Rest Of SQL}
%%%%%%%%%%%%%%%%%%%%%%%%%%%%%%%%%%%%%%%%%%%%%%%%%%%%%%%%%%%%%%%%%%%%%%%%%%%%%%%%

Other SQL features such as \emph{order by}, \emph{group by}, \emph{having} and so on can be implemented similarly:

\begin{verbatim}
  Person p = new Person();
  Alias<Integer> count = alias(count(p.id));

  List<Tuple2<Integer, Integer>> rows =
    new QueryBuilder(datasource)
      .from(p)
      .where(not(eq(p.name, constant("Peter"))))
      .groupBy(p.fatherId)
      .having(gt(count, constant(3)))
      .orderBy(desc(count))
      .select(p.fatherId, count)
      .list();
\end{verbatim}

For a native SQL fragment one can include an untyped expression. It takes the corresponding SQL string, arguments and the expression type:

\begin{verbatim}
  Person p = new Person();

  List<Tuple2<String, Integer>> rows =
    new QueryBuilder(datasource)
      .from(p)
      .select(p.name,
        unchecked(Integer.class,
          "util.count_children(id)"))
      .list();
\end{verbatim}

Implementing the full SQL is only a matter of time and Aasaru has done that as a part of his master's thesis \cite{aasaru2008}.

%%%%%%%%%%%%%%%%%%%%%%%%%%%%%%%%%%%%%%%%%%%%%%%%%%%%%%%%%%%%%%%%%%%%%%%%%%%%%%%%
%%%%%%%%%%%%%%%%%%%%%%%%%%%%%%%%%%%%%%%%%%%%%%%%%%%%%%%%%%%%%%%%%%%%%%%%%%%%%%%%
\section{Typesafe Bytecode Engineering}
%%%%%%%%%%%%%%%%%%%%%%%%%%%%%%%%%%%%%%%%%%%%%%%%%%%%%%%%%%%%%%%%%%%%%%%%%%%%%%%%
%%%%%%%%%%%%%%%%%%%%%%%%%%%%%%%%%%%%%%%%%%%%%%%%%%%%%%%%%%%%%%%%%%%%%%%%%%%%%%%%

Java bytecode is a relatively simple stack-based language. All the code is contained in methods, class structure is mostly preserved from source (fields and methods, both can be static, constructors and static initialisers are turned into special methods named \verb!<init>! and \verb!<clinit>! correspondingly). 

Inside the methods we have the variables, referred by an index with $0$ being \emph{this}, 1 being the first parameter and so on. Local variable indexes start after parameters. We can \emph{load} and \emph{store} variables. Every method has its own  stack, where we can \emph{push}, \emph{pop} and \emph{duplicate} values. We have a number of basic operations on the stack (like \emph{add} and \emph{multiply}) as well as \emph{method invocation}. When invoking the methods parameters are gathered from the stack with last parameter being on top of the stack. Finally we have some flow control, namely conditional (and unconditional) \emph{jumps}. For more information see Java Virtual Machine Specification \cite{lindholm1999jvm}.

One of the best libraries for working with Java bytecode is ASM \cite{bruneton2002acm}. It provides both a lightweight visitor-based interface and a more comfortable tree-based object-oriented interface. Unfortunately both of them (and especially visitor-based one) are completely untyped and the produced bytecode is only verified during runtime\footnote{Even that verification is quite unsatisfactory, since the JVM verifier will produce an error, but will not specify the exact place where it occurs.}.

We are going to use this simple Java example in the rest of this section:

\begin{verbatim}
public class HelloWorld {
  public static void main(String[] args) {
    System.out.println("Hello, World!");
  }
}
\end{verbatim}

When we compile it and then dump the bytecode this example becomes a bit more complicated. The following code is dumped by running \verb!javap -c HelloWorld!:

\begin{verbatim}
public class HelloWorld {
  public <init>()V
    ALOAD 0
    INVOKESPECIAL Object.<init>()V
    RETURN
  
  public static main([LString;)V
    GETSTATIC System.out : LPrintStream;
    LDC "Hello, World!"
    INVOKEVIRTUAL PrintStream.println(LString;)V
    RETURN
}
\end{verbatim}

As we can see the Java compiler has generated a default constructor for the class. Additionally all classes are referred by their full names\footnote{In this example and further we omit the package from the class name for brevity}. The instructions we see do the following:
\begin{itemize}
\item \verb!ALOAD! loads the local variables to the top of the stack, index $0$ is \emph{this}.
\item \verb!INVOKESPECIAL! in this case invokes super method consuming an \verb!Object! from the stack.
\item \verb!GETSTATIC! retrieves the value of the static fields and puts it on the stack (in this case a \verb!PrintStream!).
\item \verb!LDC! pushes a constant value to the stack.
\item \verb!INVOKEVIRTUAL! invokes a usual (virtual) method, consuming the parameters from the stack and pushing the result to the stack.
\item \verb!RETURN! exits from the method
\end{itemize}

We are not going to examine ASM library in detail, suffice to say its API corresponds almost one-to-one to Java bytecode instructions.

%%%%%%%%%%%%%%%%%%%%%%%%%%%%%%%%%%%%%%%%%%%%%%%%%%%%%%%%%%%%%%%%%%%%%%%%%%%%%%%%
\subsection{Typesafe DSL}
%%%%%%%%%%%%%%%%%%%%%%%%%%%%%%%%%%%%%%%%%%%%%%%%%%%%%%%%%%%%%%%%%%%%%%%%%%%%%%%%

Let's ignore the default constructor for a second and concentrate on the \verb!main()! method. The typesafe DSL that we propose would engineer this bytecode like this:

\begin{verbatim}
new ClassBuilder(
  cw, V1_4, ACC_PUBLIC, "HelloWorld", "Object", null)    
  .beginStaticMethod(
    ACC_PUBLIC | ACC_STATIC, 
    "main", void.class, String[].class)
  .getStatic(System.class, "out", PrintStream.class)
  .push("Hello, World!")
  .invokeVirtualVoid(
    PrintStream.class, "println", String.class)
  .returnVoid()
  .endMethod();
\end{verbatim}

What do we mean under typesafe in this case? Well in addition to most parameters being passed as class literals and the instructions syntax being part of the API we also track stack and local variable types. For example if we exchange the \verb!push("Hello, World!")! to \verb!push(10)! the compiler will give the following error:

\begin{verbatim}
invokeVirtualVoid(..., Class<? super Integer>) 
in MethodBuilderS2V1<PrintStream, Integer, String[]>
is not applicable to (..., Class<String>)
\end{verbatim}

The error means that our DSL tracks the types of the stack slots and since the method expects a \verb!String! parameter or compatible and the stack contains an \verb!Integer! compiler produces an error. 

%%%%%%%%%%%%%%%%%%%%%%%%%%%%%%%%%%%%%%%%%%%%%%%%%%%%%%%%%%%%%%%%%%%%%%%%%%%%%%%%
\subsection{Tracking Stack and Local Variables}
%%%%%%%%%%%%%%%%%%%%%%%%%%%%%%%%%%%%%%%%%%%%%%%%%%%%%%%%%%%%%%%%%%%%%%%%%%%%%%%%

To achieve type safety we need to track stack size and types. We wouldn't want to allow to \verb!pop()! off an empty stack. And we would want to ensure that when you invoke a method all the parameters are there. 

Since there is a different set of operations allowed for different stack sizes it is natural to have a class for each stack size. We choose to name (and number them) \verb!MethodBuilderS0!, \verb!MethodBuilderS1!, \verb!MethodBuilderS2! and so on. Each of them has only the methods possible with the current stack size. Now the method for \verb!pop()! will not even show up in autocompletion if the stack is not large enough.

We can apply the same idea to tracking variables, allowing to add only one variable at a time and providing methods \verb!loadVar*! and \verb!stroreVar*!. This way our class name becomes \verb!MethodBuilderS*V*! where $S$ stands for stack size and $V$ stands for variable count.

Of course in addition to the sizes we also have to track the actual types. To do that each \verb!MethodBuilder! is parametrized by $N$ type variables where $N = S + V$. The \verb!MethodBuilderS2V1<PrintStream, Integer, String[]>! we saw previously in the compiler output means that two stack slots have types \verb!PrintStream! and \verb!Integer!, whereas the only local variable has type \verb!String[]!.

To understand how the types are inferred let's see the implementation of \verb!push()! and \verb!pop()!:

\begin{verbatim}
public class MethodBuilderS2V1 <S0, S1, V0> {
  public MethodBuilderS1V1<S0, V0> pop() {
    ...
    return 
      new MethodBuilderS1V1<S0, V0>(cb, mv);
  }

  public <S> MethodBuilderS3V1<S0, S1, S, V0> 
    push(S value) {
    ...
    return 
      new MethodBuilderS3V1<S0, S1, S, V0>(cb, mv);
  }
\end{verbatim}

The class in question, \verb!MethodBuilderS2V1! is parametrised by two stack types and one variable types. The method \verb!pop()! throws away the top stack type and returns an instance of \verb!MethodBuilderS1V1!. The method push on the other hand infers an addition type from the parameter passed to it and returns an instance of \verb!MethodBuilderS3V1! adding the inferred type to the top of the stack types.

What does the \verb!invokeVirtualVoid()! look like? First of all, it needs to consume two stack variables, therefore we need at least \verb!MethodBuilderS2V*! class to call it. Therefore all classes with less stack variables will not have this method. Secondly we need to check that the types in the stack are fitting, but must allow some leniency due to subtyping:

\begin{verbatim}
public class MethodBuilderS2V1<S0, S1, V0> {
  public MethodBuilderS0V1<V0> 
    invokeVirtualVoid(
      Class< ? super S0> owner, 
      String name, 
      Class< ? super S1> parameter1) {
    ...    
    return new MethodBuilderS0V1<V0>(cb, mv);
  }
}
\end{verbatim}

As you can see it indeed consumes two stack values returning \verb!MethodBuilderS0V1!. If our method would also return a result we would need to pass the result type as well, and return a class with stack depth only one less. The expression \verb!? super S0! means that we require the actual parameter type to be a superclass of the stack type, which provides the required leniency.

%%%%%%%%%%%%%%%%%%%%%%%%%%%%%%%%%%%%%%%%%%%%%%%%%%%%%%%%%%%%%%%%%%%%%%%%%%%%%%%%
\subsection{Unsafe Assumptions}
%%%%%%%%%%%%%%%%%%%%%%%%%%%%%%%%%%%%%%%%%%%%%%%%%%%%%%%%%%%%%%%%%%%%%%%%%%%%%%%%

One of the problems with the DSL we proposed here is that it assumes all the types exist. However it is very often the case that some of the types (most prominently the class currently being created) do not. You can somewhat alleviate the problem by introducing a special placeholder \verb!Self! type and use it instead of the current class name, but it won't solve the problem of other classes still awaiting construction.

A different (but connected) problem is that you are not always constructing the full method, instead you could be just creating a prelude for a particular method or replacing one instruction with a series of your own. To solve this we should allow escaping from the rigid typesafe world by having unsafe operations, which issue compiler warnings.

This means that instead of missing \verb!pop()! from a type with no stack slots we should just deprecate it or otherwise issue a warning. This also means that we should have \verb!invoke*()! methods that take strings as parameter types, similarly deprecated.

However, since we know it's not a perfect world we'd like to at least protect ourselves a bit better. Therefore we should allow users to document their assumptions.

This means that when we need to write just a fragment of bytecode we want to document what stack values and variables will it need. For that we introduce methods \verb!assumePush()!/\verb!assumePop()! and \verb!assumeVar*()!, which do not push any values, but just add the corresponding type variables:

\begin{verbatim}
public class MethodBuilderS2V1<S0, S1, V0> {
  public <S> MethodBuilderS3V1<S0, S1, S, V0> 
      assumePush(Class<S> type) {
    return 
      new MethodBuilderS3V1<S0, S1, S, V0>(cb, mv);
  }

  public MethodBuilderS1V1<S0, V0> assumePop() {
    return new MethodBuilderS1V1<S0, V0>(cb, mv);
  }

  public <V> MethodBuilderS2V2<S0, S1, V0, V> 
      assumeVar1(Class<V> type) {
    return 
      new MethodBuilderS2V2<S0, S1, V0, V>(cb, mv);
  }  
}	
\end{verbatim}

Using them we can document our expectations and let the compiler validate them. The following is an example of how we can use the assumptions to document that we expect \verb!PrintStream! to be on stack and \verb!String[]! to be the variable $0$.

\begin{verbatim}
private static void 
    genSayHello(MethodBuilderS0V0 mb) {
  mb.assumeVar0(String[].class)
  .assumePush(PrintStream.class)
  .loadVar0(String[].class)
  .push(0)
  .arrayLoad(
    String[].class, 
    Integer.class,
    String.class)
  .invokeVirtualVoid(
    INVOKEVIRTUAL, 
    PrintStream.class, 
    "println", 
    String.class);
}
\end{verbatim}

%%%%%%%%%%%%%%%%%%%%%%%%%%%%%%%%%%%%%%%%%%%%%%%%%%%%%%%%%%%%%%%%%%%%%%%%%%%%%%%%
\subsection{Control Flow and Reuse}
%%%%%%%%%%%%%%%%%%%%%%%%%%%%%%%%%%%%%%%%%%%%%%%%%%%%%%%%%%%%%%%%%%%%%%%%%%%%%%%%

Similar as in SQL case we introduce closures to deal with control flow and reuse. The problem is aggravated in this case as we return a different type every time a method is called. We introduce a closure type as before:

\begin{verbatim}
public interface Closure {
  public void apply(MethodBuilderS0V0 mb);
}

public class MethodBuilderS2V1 <S0, S1, V0> {
  public MethodBuilderS2V1<S0, S1, V0> 
      closure(Closure closure) {
    closure.apply(new MethodBuilderS0V0(cb, mv));
    return this;
  }
}
\end{verbatim}

We illustrate using closures by calling the method \verb!genSayHello()! introduced previously:

\begin{verbatim}
.beginStaticMethod(
  ACC_PUBLIC | ACC_STATIC, 
  "main", void.class, String[].class)
.getStatic(System.class, "out", PrintStream.class)
.closure(new Closure() {
  public void apply(MethodBuilderS0V0 mb) {
    genSayHello(mb);
  }
})
.returnVoid()
.endMethod();
\end{verbatim}

Of course with the introduction of actual closures in Java 7 this would look much shorter.

%%%%%%%%%%%%%%%%%%%%%%%%%%%%%%%%%%%%%%%%%%%%%%%%%%%%%%%%%%%%%%%%%%%%%%%%%%%%%%%%
%%%%%%%%%%%%%%%%%%%%%%%%%%%%%%%%%%%%%%%%%%%%%%%%%%%%%%%%%%%%%%%%%%%%%%%%%%%%%%%%
\section{Patterns and Discussion}
%%%%%%%%%%%%%%%%%%%%%%%%%%%%%%%%%%%%%%%%%%%%%%%%%%%%%%%%%%%%%%%%%%%%%%%%%%%%%%%%
%%%%%%%%%%%%%%%%%%%%%%%%%%%%%%%%%%%%%%%%%%%%%%%%%%%%%%%%%%%%%%%%%%%%%%%%%%%%%%%%

Let us now take a step back and look at the patterns showing up in the design of the two DSLs we have introduced. We will try to formulate a one-sentence summary for each of the patterns that we have identified and then follow it with examples and informal discussion. Although we could have put them down in a more formal way as in Design Patterns \cite{gamma1995dpe}, we find that the issues are too broad and the DSL design too much of an art to assume such formality.

Note that as previously in text we refer to the classes that implement the DSL API as \emph{builders}. 

%%%%%%%%%%%%%%%%%%%%%%%%%%%%%%%%%%%%%%%%%%%%%%%%%%%%%%%%%%%%%%%%%%%%%%%%%%%%%%%%
\subsection{Restricting Syntax}
%%%%%%%%%%%%%%%%%%%%%%%%%%%%%%%%%%%%%%%%%%%%%%%%%%%%%%%%%%%%%%%%%%%%%%%%%%%%%%%%

\begin{quote}
\emph{At any moment of time the DSL builder should have precisely the methods allowed in the current state.}
\end{quote}

We saw several examples of this pattern at work
\begin{itemize}
\item SQL query builders allowed \emph{from}, \emph{where} and \emph{select} to be called once and only once.
\item Bytecode builders hide methods that consume more stack slots than is available.
\end{itemize}

This pattern was discussed in a different setting in \cite{freeman2004jsr}. They used the interfaces to encode the syntax of the DSL while still implementing all of the interfaces by a few classes. In our case such an approach would work in the SQL case study language, but not in the bytecode engineering case study language, as we would still need method with different return types depending on the amount of tracked stack slots and local variables. 

This is the first principle for building a typesafe DSL as it allows to encode the base syntax in a typesafe manner.

%%%%%%%%%%%%%%%%%%%%%%%%%%%%%%%%%%%%%%%%%%%%%%%%%%%%%%%%%%%%%%%%%%%%%%%%%%%%%%%%
\subsection{Type History}
%%%%%%%%%%%%%%%%%%%%%%%%%%%%%%%%%%%%%%%%%%%%%%%%%%%%%%%%%%%%%%%%%%%%%%%%%%%%%%%%

\begin{quote}
\emph{You can accumulate type history as a type list and use it to reject actions that do not fit with that history}
\end{quote}

We saw several examples of this pattern at work
\begin{itemize}
\item Select builders types encode information about selected columns. 
\item Bytecode builders encode information about stack slot and local variable types.
\item Bytecode builder methods that consume stack slots must consume types fitting to the ones currently on stack.
\item Bytecode builder local variable methods infer types from the values to variables and from variables to stack.
\end{itemize}

Type history is an important concept as it allows to reject actions based not on the types in the current method call, but in the previous method calls. In some DSLs type history comes naturally and makes up the basis of the language, like in case of the bytecode engineering DSL introduce previously. We conjecture that in any language that uses either stack or a stack-like environment this is the best approach to use.

%%%%%%%%%%%%%%%%%%%%%%%%%%%%%%%%%%%%%%%%%%%%%%%%%%%%%%%%%%%%%%%%%%%%%%%%%%%%%%%%
\subsection{Typesafe Metadata}
%%%%%%%%%%%%%%%%%%%%%%%%%%%%%%%%%%%%%%%%%%%%%%%%%%%%%%%%%%%%%%%%%%%%%%%%%%%%%%%%

\begin{quote}
\emph{Metadata used by your DSL should include compile-time type information.}
\end{quote}

Both our DSLs made use of typesafe metadata, some of those uses were less obvious than others:
\begin{itemize}
\item SQL made use of pregenerated metadata dictionary that contained type information about database objects including tables and columns.
\item The SQL expressions and named aliases provided information about the type of expression. This metadata was accessible via the local variable that the expression was saved to.
\item The SQL \verb!SelectBuilder! encoded metadata about its column types, which would be important if we wanted to use it as a subquery again saved in a variable.
\item The bytecode DSL used class literals, which are typesafe metadata embedded in Java language.
\end{itemize}

There are two type of typesafe metadata we can separate: metadata dictionary and runtime metadata. 

Metadata dictionary is the metadata that is either generated or otherwise available in a static manner. This is the main building block of the DSLs that need to be extended beyond their basic syntax. 

However we also need to extend the dictionary and that is where the runtime metadata comes in. The runtime metadata is encoded as a runtime expression with a static type. The type must encode all the metadata that we need. This way we can introduce local variables referring to such expressions on demand extending the metadata dictionary while still retaining the type safety property

%%%%%%%%%%%%%%%%%%%%%%%%%%%%%%%%%%%%%%%%%%%%%%%%%%%%%%%%%%%%%%%%%%%%%%%%%%%%%%%%
\subsection{Unsafe Assumptions}
%%%%%%%%%%%%%%%%%%%%%%%%%%%%%%%%%%%%%%%%%%%%%%%%%%%%%%%%%%%%%%%%%%%%%%%%%%%%%%%%

\begin{quote}
\emph{Allow the user to do type unsafe actions, but make sure he has to document his assumptions.}
\end{quote}

We saw two example of this pattern:
\begin{itemize}
\item In SQL the unchecked expressions still had to declare the expected type of the expression.
\item The bytecode DSL allowed ignoring the type history by providing dedicated methods for type unsafe actions.
\item In bytecode DSL we saw \verb!assume*! methods dedicated to documenting type assumptions.
\end{itemize}

Both parts of this pattern are very important. If you don't allow type unsafe actions it is likely that the users will find it limiting at some point and revert to bypassing your DSL. However using two APIs instead of one makes the code harder to read and maintain.

If you don't allow users to document their assumptions you risk errors every time you do something unsafely. Assumptions allow to always hold the information about types in a local manner, even though it does not guarantee full type safety it still helps to develop safer code.

%%%%%%%%%%%%%%%%%%%%%%%%%%%%%%%%%%%%%%%%%%%%%%%%%%%%%%%%%%%%%%%%%%%%%%%%%%%%%%%%
\subsection{Hierarchical Expressions}
%%%%%%%%%%%%%%%%%%%%%%%%%%%%%%%%%%%%%%%%%%%%%%%%%%%%%%%%%%%%%%%%%%%%%%%%%%%%%%%%

\begin{quote}
\emph{Use method chaining when you need context and static functions when you need hierarchy and extensibility.}
\end{quote}

One of the main problems with method chaining is very poor support for hierarchy. We could have also structured the SQL like this:
\begin{verbatim}
//...
.where
  .and()
    .eq(p.id, 3)
    .gt(p.age, 17)
  .endAnd()
//...  
\end{verbatim}
However it is clumsy and the first time we would auto-format the source code we would lose all hints of hierarchy in the code structure.

The other problem with method chaining is extensibility. In the case we just shown \verb!and()! has to return a builder that includes all of the SQL expressions (including the \verb!and()! itself). However different databases have different expressions and we need different sets of expressions in different cases.

The obvious solution in this case is using static functions (and \verb!import static! to bring them into the local namespace). However static functions suffer from a different problem---unlike chained methods they cannot access the context. The context includes all the state that was accumulated by the previous chained method calls as well as type history. 

On the other hand static functions can be added on demand, whereas builders are much harder to extend. In the end it is likely that you will need both for different parts of your language. In the case of SQL we chose to make the statement-like part of the syntax chained and the expression-like part of the system functional.

%%%%%%%%%%%%%%%%%%%%%%%%%%%%%%%%%%%%%%%%%%%%%%%%%%%%%%%%%%%%%%%%%%%%%%%%%%%%%%%%
\subsection{Closures}
%%%%%%%%%%%%%%%%%%%%%%%%%%%%%%%%%%%%%%%%%%%%%%%%%%%%%%%%%%%%%%%%%%%%%%%%%%%%%%%%

\begin{quote}
\emph{Use closures to escape the method chaining for control flow and reuse.}
\end{quote}

We introduced closures in both DSLs that we reviewed. They allow for a flexible way to mix your DSL with the control flow of the host language. Of course if Java would have support for true closures it could be made more concise, but even with the current syntax it is quite usable.

%%%%%%%%%%%%%%%%%%%%%%%%%%%%%%%%%%%%%%%%%%%%%%%%%%%%%%%%%%%%%%%%%%%%%%%%%%%%%%%%
%%%%%%%%%%%%%%%%%%%%%%%%%%%%%%%%%%%%%%%%%%%%%%%%%%%%%%%%%%%%%%%%%%%%%%%%%%%%%%%%
%\section{Type Safety Limitations}
%%%%%%%%%%%%%%%%%%%%%%%%%%%%%%%%%%%%%%%%%%%%%%%%%%%%%%%%%%%%%%%%%%%%%%%%%%%%%%%%
%%%%%%%%%%%%%%%%%%%%%%%%%%%%%%%%%%%%%%%%%%%%%%%%%%%%%%%%%%%%%%%%%%%%%%%%%%%%%%%%

%\textbf{TODO}

%%%%%%%%%%%%%%%%%%%%%%%%%%%%%%%%%%%%%%%%%%%%%%%%%%%%%%%%%%%%%%%%%%%%%%%%%%%%%%%%
%%%%%%%%%%%%%%%%%%%%%%%%%%%%%%%%%%%%%%%%%%%%%%%%%%%%%%%%%%%%%%%%%%%%%%%%%%%%%%%%
\section{Acknowledgements}
%%%%%%%%%%%%%%%%%%%%%%%%%%%%%%%%%%%%%%%%%%%%%%%%%%%%%%%%%%%%%%%%%%%%%%%%%%%%%%%%
%%%%%%%%%%%%%%%%%%%%%%%%%%%%%%%%%%%%%%%%%%%%%%%%%%%%%%%%%%%%%%%%%%%%%%%%%%%%%%%%

We would like to thank Juhan Aasaru, who developed the SQL DSL as a part of his master's thesis \cite{aasaru2008}. He did a great job and contributed a lot of ideas that influenced this article. 

We would also like to thank the readers of the dow.ngra.de blog that pointed out some of the problems with our initial prototypes.

This work was partially supported by Estonian Science Foundation grant No. ????. 

%%%%%%%%%%%%%%%%%%%%%%%%%%%%%%%%%%%%%%%%%%%%%%%%%%%%%%%%%%%%%%%%%%%%%%%%%%%%%%%%
%%%%%%%%%%%%%%%%%%%%%%%%%%%%%%%%%%%%%%%%%%%%%%%%%%%%%%%%%%%%%%%%%%%%%%%%%%%%%%%%
\section{Conclusions and Future Work}
%%%%%%%%%%%%%%%%%%%%%%%%%%%%%%%%%%%%%%%%%%%%%%%%%%%%%%%%%%%%%%%%%%%%%%%%%%%%%%%%
%%%%%%%%%%%%%%%%%%%%%%%%%%%%%%%%%%%%%%%%%%%%%%%%%%%%%%%%%%%%%%%%%%%%%%%%%%%%%%%%

In this paper we have introduced two different domain specific languages, trying to use the most of Java type system to inhibit wrong behaviour at compile-time. Although the languages were from two completely different domains we managed to identify a number of general patterns, which we conjecture to be domain-independent and which will ease the work of future DSL designers.

Unlike most previous Java DSLs we found that mixing the Fluent Interface idiom with static functions, metadata and closures provides a better experience to the users than pure method chaining. This also provides for easier extension of the DSL and easier reuse of the DSL fragments. 

We made very liberal use of the Java 5 Generics to improve the type safety properties of the DSLs. We managed to secure both case study DSLs against almost all type mistakes a programmer can make. To do that we introduced type lists that allowed us to verify chained method calls against the history of calls. We also found that it is essential to allow user to both escape the type safety limits and document the assumptions so that the rest of the DSL fragment could still be typesafe.

We plan to continue investigating the limits of typesafe embedded DSLs in Java foremost by implementing new case studies and extracting patterns and idioms that of less generic manner. One of the developments we plan to pursue at the moment is a context-aware DSL for building hierarchical data in Java. There will definitely be others as well.

\bibliographystyle{abbrv}
\bibliography{kabanov-raudjarv-pppj08}

\end{document}
